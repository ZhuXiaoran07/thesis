
\chapter{基本概念和预备知识}



\section{并发操作系统}
并发操作系统是指在只有一个处理器工作的情况下,同一时间段可以有多个进程在运行,但是任一时刻点最多只能有一个进程可以运行。并发操作系统的可以充分利用计算机的所有资源,像cpu,外围设备、内存等。一个进程在执行时很大程度上在等待资源,而进程等待资源时,我们不希望它仍然占用CPU,而是可以把CPU释放给其它可以运行的进程。
\section{可达性逻辑}
可达性逻辑(Reachability Logic)由美国伊利诺伊大学厄巴纳香槟分校Grigore Rosu教授所在的团队提出,其目的是:
\begin{itemize}
	\item 构建了一套语言无关的用来验证系统可达性的证明系统;
	\item 所定义的语言操作语义可用作程序证明时所用的公理,从而减小做程序证明的工作量。
\end{itemize}
典型的,霍尔三元组
\section{精化关系定义}
\section{K 框架}
\section{AUTOSAR规范}


\section{本章小结}

\label{ch2}



