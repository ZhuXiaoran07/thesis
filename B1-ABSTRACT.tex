\vspace{-2.5cm}
\chapter*{\zihao{2}\heiti{摘~~~~要}}
\vspace{-1cm}

\setlength{\baselineskip}{25pt}
软件可信性是建立在可靠性、安全性、可维护性等众多软件属性上,用于衡量软件运行过程与结果符合用户预期程度的综合属性。由于软件事故频发,人们对软件可信性的关注越来越多,如何准确高效的预测软件可信性是研究软件可信性的关键问题。相比于评估软件开发过程来度量软件可信性,对软件实体可信性的预测更加贴合用户实际需求。软件是由程序和文档构成的,其中程序是软件运行的主体。本文基于对软件源代码的可信性预测来反映软件可信性,使得度量结果更加准确高效。

首先,本文首先给出软件失信及程序中的失信证据的概念,从重大软件事故中收集典型的失信证据,并对失信证据进行分类。其次,借鉴基于属性的软件可信度量体系中可信性分级评估模型,给出适用于失信证据的可信分级模型,该模型确定了每条失信证据的可信度。

失信证据

接着,基于失信证据的可信度和失信证据对整个软件程序的影响,引入信息熵的概念,在信息熵模型基础上建立软件可信性预测模型。

再次,给出确定该预测模型中失信证据的特征参数的方法。

再次,基于失信证据和静态检测工具CppCheck进行二次开发,开发静态预测软件可信性工具。该工具不仅提供对C/C++程序的典型缺陷检测,还提供对除缺陷以外的失信证据的检测,根据模型给出失信证据的可信度,并计算出当前软件的可信度,根据可信性分级模型给出当前软件的可信性。其中,检测出的失信证据和计算后的结果均可视化呈现。

最后,

\hspace{-0.5cm}
\sihao{\heiti{关键词:}} \xiaosi{软件可信性,静态预测,失信证据,信息熵,CppCheck,缺陷检测}
